\documentclass[%
12pt, %
final, % 
oneside, % 
onecolumn, %  
centertags]{article} % относится к классу article и размер шрифта 12 пунктовб, {article: статья, report: отчеты и диссертации, book: книга, letter: письмо}

% \usepackage{fontspec}
 
% \setmainfont{Times New Roman}

% \documentclass[a4paper, 12pt]{report}

\topmargin= -30pt % насколько сверху будет страница
\textheight= 650pt


\usepackage[utf8]{inputenc} % задает кодировку, utf-8 кодировка, включающая в себя знаки почти всех языков мира
\usepackage[english]{babel} % подключает необходимые языки, основным языком является английский

\selectlanguage{english} % настройки будут на английском, но писать будет на русском

\usepackage{euscript}
\usepackage{supertabular}

\renewcommand{\baselinestretch}{1.0} 

\usepackage[colorlinks=true,linkcolor=blue,unicode=true,urlcolor = blue]{hyperref} %hypered
\usepackage[pdftex]{graphicx} % для графики

\usepackage{amsthm, amssymb, amsmath, amsfonts} % математический пакет, математические шрифты
\usepackage{textcomp}
\usepackage[noend]{algorithmic}
\usepackage[ruled]{algorithm}
\usepackage{lipsum}
\usepackage{indentfirst}
\usepackage{babel}
\usepackage{pgfplots}
\usepackage{setspace}
\usepackage{xcolor}
\usepackage{hyperref}
\usepackage{subfigure}

\setcounter{secnumdepth}{5}
\setcounter{tocdepth}{5}
\newcommand\simpleparagraph[1]{%
  \stepcounter{paragraph}\paragraph*{\theparagraph\quad{}#1}}
\usepackage{listings}
% \usepackage{xcolor}
%\usepackage{minted}

\lstset { %
     language=C++,
     backgroundcolor=\color{black!5}, % set backgroundcolor
     basicstyle=\footnotesize,% basic font setting
}


\linespread{1.0} 
\setlength{\parindent}{2.4em}
\setlength{\parskip}{0.1em}

\pgfplotsset{compat=1.9}
\pgfplotsset{model/.style = {blue, samples = 100}} 
\pgfplotsset{experiment/.style = {red}}

\theoremstyle{plain}
\binoppenalty=10000

\newtheorem{theorem}{Theorem}[section] % theorem

\theoremstyle{definition}
% \newtheorem{definition}{Определение}[subsection]
\newtheorem{definition}{Definition}[subsection]

\theoremstyle{remark}
% \newtheorem{remark}{Замечание}[section]

% \newtheorem{corollary}{Следствие}

% \newtheorem{proposition}{Proposition}

% \newtheorem{example}{Пример}

% \newtheorem{lemma}{Лемма}[section]

\renewcommand*{\proofname}{Proof}

\graphicspath{ {./images/} }


% \usepackage{amsmath,amsfonts,amssymb, setspace}  % Разнообразные математические команды и значки
% \usepackage{indentfirst}     % Отступ в первом абзаце

% \pagestyle{empty}
\usepackage[left=2.5cm, right=1.5cm, top=2.5cm, bottom=2.5cm]{geometry}
\usepackage[medium]{titlesec}
\usepackage{graphicx}
% \graphicspath{ {./images/} }

\begin{document}

	\begin{titlepage} 
		\begin{center}
		\textbf{}\\[2.0cm]
		\LARGE FEDERAL STATE AUTONOMOUS EDUCATIONAL INSTITUTION OF HIGHER EDUCATION \\[0.5cm]
		\Large ITMO UNIVERSITY \\[3cm]
		\LARGE Report\\
		\Large MPI. Assignments $3-5$ \\
		\Large Parallel algorithms for the analysis and synthesis of data \\[4cm]


		\begin{flushright}
		Performed by\\
		Aleksandr Shirokov\\
		J4133c\\
		Accepted by\\
		Petr Andriushchenko

		Deadline: 19.12.21
		\end{flushright}

		\vfill 

		{\Large {St. Petersburg}} \par
		{\Large {2021}}
		\end{center} 
	\end{titlepage}

\tableofcontents
\newpage


\section{Assignments}

\subsection{Assignment 3.}

\subsubsection{Formulation of the problem}

Compile and run \textsc{Assignment3.c} program. Explain in detail how it works.

\subsubsection{Example of launch parameters and output. Detailed description of solution}

Code for \textbf{assignment 3} is \href{https:\//github.com/aptmess/parallel_algorithms/blob/master/HT/hw_mpi/Assignment3.c}{here}.

Compilation example: \textsc{mpic++ -o ./cpf/3.o Assignment3.c}

Launch example:

\begin{itemize}
	\item Not in parallel: \textsc{./cpf/3.o}

	\begin{center}
		\includegraphics[scale=0.5]{3.1.png}
	\end{center}


	\item In parallel: \textsc{mpirun --oversubscribe -np 4 ./cpf/3.o} (have a problem without \textsc{--oversubscribe} option.

	\begin{center}
		\includegraphics[scale=0.5]{3.2.png}
	\end{center}

\end{itemize}

Let's move to the the code and explain how it works.

\begin{center}
		\includegraphics[scale=0.75]{3.code.png}

		Assignment3 code
\end{center}

\begin{enumerate}
	\item Line 5 - \textsc{MPI\_Init} - initialisation, starting the parallel part with arguments of main function;
	\item Line 6 - initialize variables, especially \textit{rank} for rank of process and \textit{n} for amount of process
	\item Line 7 - creating \textsc{MPI\_status}, variable status contains three attibutes of message:
	\begin{itemize}
		\item \textsc{MPI\_source} - number of the sending process;
		\item \textsc{MPI\_tag} - name of text message, identifier;
		\item \textsc{MPI\_error} - error code.
	\end{itemize}
	\item Line 7 - getting number of processes (\textit{n} varialbe)
	\item Line 8 - getting rank of the process (\textit{rank} variable)
	\item Line 10 - \textsc{if} statement
	\begin{itemize}
		\item \textsc{if rank == 0} - then we are printing 'Hello from process 0' and in range $i=[1..n]$ where $i$ is number of process, waiting for incoming messages from any source (who is quicklier will be write earlier) with any tag using syntax of function \textsc{MPI\_Send}. The information 'Hello from process \%message' is printing after we get message on each iteration of cycle.
		\item else - from processes with rank not $0$ sending messages to process with rank $0$. The message contains information about processes's rank.
	\end{itemize}
	\item Line 19 - Ending parallel part
\end{enumerate}

I have analysed the first code with mpi, explain how it works, compile it and show results.

\newpage
\subsection{Assignment 4.}

\subsubsection{Formulation of the problem}

\begin{enumerate}
	\item Convert the code \textsc{Assignment4.c} to match your individual version of the assignment.
	\item My task according to the \href{https://docs.google.com/spreadsheets/d/1m2KqmSc5pMMQ9JezZGybnuKG75FywUcs8gbPuMvVUxc/edit#gid=1272007391}{table} is $27$.
	\item If the number of processes in the task is not defined, then we can assume that this number does not exceed 8. \item The \textit{main process} is understood as a process of rank $0$ for the communicator \textsc{MPI\_COMM\_WORLD}. \item For all processes of nonzero rank in the tasks, the common name of the subordinate processes is used. 
	\item If the task does not define the maximum size of a set of numbers, then it should be considered equal to 10.
\end{enumerate}

\textbf{Variant $27$}:

\begin{enumerate}
	\item Each subprocess is given four integers. 
	\item Send these numbers to the main process using one call to the \textsc{MPI\_Send} function for each sending process, and display them in the main process. 
	\item The resulting numbers should be displayed in ascending order of the ranks of the processes that sent them.
\end{enumerate}


\subsubsection{Example of launch parameters and output. Detailed description of solution}

Code for \textbf{assignment 4} is \href{https:\//github.com/aptmess/parallel_algorithms/blob/master/HT/hw_mpi/Assignment4.c}{here}.

Compilation example: \textsc{mpic++ -o ./cpf/4.o Assignment4.c}

Launch example:\textsc{mpirun --oversubscribe -np 8 ./cpf/4.o} (have a problem without \textsc{--oversubscribe} option.

\begin{center}
\includegraphics[scale=0.55]{4.1.png}

Works correctly - four integers in each subprocess was send to main process and displayed in ascending order of the ranks
\end{center}

Let's move to the the code and explain how it works.

\begin{center}
		\includegraphics[scale=0.75]{4.code.png}

		Assignment4 code
\end{center}

\begin{enumerate}
	\item Line 7 - \textsc{MPI\_Init} - initialisation, starting the parallel part with arguments of main function;
	\item Line 8 - initialize variables, especially \textit{rank} for rank of process and \textit{n} for amount of process
	\item Line 9 - initializing int array of size 4 for input messages from subprocesses;
	\item Line 10 - creating \textsc{MPI\_status}, variable status contains three attibutes of message:
	\begin{itemize}
		\item \textsc{MPI\_source} - number of the sending process;
		\item \textsc{MPI\_tag} - name of text message, identifier;
		\item \textsc{MPI\_error} - error code.
	\end{itemize}
	\item Line 11 - getting number of processes (\textit{n} varialbe)
	\item Line 12 - getting rank of the process (\textit{rank} variable)
	\item Line 13 - \textsc{if} statement
	\begin{itemize}
		\item \textsc{if rank == 0} - then we are printing 'Start main process 0' and in range $i=[1..n]$ where $i$ is number of process, waiting for incoming messages from each $i$ source (from $1$ to $n$) with any tag using syntax of function \textsc{MPI\_Send}. When message is recieved, the array of $5$ integers is displayed in process 0 from process $i$ - first $4$ is integers that we need to display and the last - rank of process (something like check that program works correctly). Using this algrotihm we print results of process in ascending order of the ranks.
		\item else - from processes with rank not $0$ sending messages to process with rank $0$. Generate int array of size $5$ with different number for each process using line $29$ - first $4$ elements are random integers from 1to 20 and the last is rank of process. The message contains information about generated array using interface \textsc{MPI\_send} - start with first element of array \textsc{sending}, $4$ elements of size int to process $0$.
	\end{itemize}
	\item Line 19 - Ending parallel part
\end{enumerate}

I have implemented an algorithm from task and display results of processes in ascending order.

\newpage
\subsection{Assignment 5.}

\subsubsection{Formulation of the problem}

\begin{enumerate}
	\item Compile and run \textsc{Assignment5.c} program.
	\item Explain in detail how it works.
	\item Determine the execution time of the program from the previous task.
\end{enumerate}

Task from previous lab: variant $27$.

\subsubsection{Example of launch parameters and output. Detailed description of solution}

\paragraph{Part I. Explain Assignment 5.c}

$\$

Code for \textbf{assignment 5} is \href{https:\//github.com/aptmess/parallel_algorithms/blob/master/HT/hw_mpi/Assignment5.c}{here}.

Compilation example: \textsc{mpic++ -o ./cpf/5.o Assignment5.c}

Launch example:\textsc{mpirun --oversubscribe -np 8 ./cpf/5.o} (have a problem without \textsc{--oversubscribe} option.

\begin{center}
\includegraphics[scale=0.55]{5.1.png}
\end{center}

Let's move to the the code and explain how it works.

\begin{center}
		\includegraphics[scale=0.75]{5.1.code.png}

		Assignment5 code
\end{center}

The main purpose of the problem is to count average time consumption of every process. This program takes as an input parameter an amount of processes, for \textsc{NTIMES} run the program and count the average time of each process works, also the processor and the rank of process is displayed.

\paragraph{Part II. Determine the execution time of the program from the previous task}

$\ $

Code for \textbf{assignment 5 partII} is \href{https:\//github.com/aptmess/parallel_algorithms/blob/master/HT/hw_mpi/5.1.c}{here}.

Compilation example: \textsc{mpic++ -o ./cpf/5.1.o 5.1.c}

Launch example:\textsc{mpirun --oversubscribe -np 8 ./cpf/5.1.o} (have a problem without \textsc{--oversubscribe} option.

\begin{center}
\includegraphics[scale=0.55]{5.2.png}
\end{center}

Let's move to the the code and explain how it works.

\begin{center}
\includegraphics[scale=0.7]{5.2.code.png}
Assignment5 part II code
\end{center}

I've implemented counting working time of each process as it was implemented in \textsc{Assignment5.c}. As we expected, the working time of process $0$ is the biggiest, because main process waiting for other processes's results.



\subsection{Appendix}

The link to the sourse code which is placed on my \href{https://github.com/aptmess/parallel_algorithms}{github}.


\end{document}