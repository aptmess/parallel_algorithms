\documentclass[%
12pt, %
final, % 
oneside, % 
onecolumn, %  
centertags]{article} % относится к классу article и размер шрифта 12 пунктовб, {article: статья, report: отчеты и диссертации, book: книга, letter: письмо}

% \usepackage{fontspec}
 
% \setmainfont{Times New Roman}

% \documentclass[a4paper, 12pt]{report}

\topmargin= -30pt % насколько сверху будет страница
\textheight= 650pt


\usepackage[utf8]{inputenc} % задает кодировку, utf-8 кодировка, включающая в себя знаки почти всех языков мира
\usepackage[english]{babel} % подключает необходимые языки, основным языком является английский

\selectlanguage{english} % настройки будут на английском, но писать будет на русском

\usepackage{euscript}
\usepackage{supertabular}

\renewcommand{\baselinestretch}{1.0} 

\usepackage[colorlinks=true,linkcolor=blue,unicode=true,urlcolor = blue]{hyperref} %hypered
\usepackage[pdftex]{graphicx} % для графики

\usepackage{amsthm, amssymb, amsmath, amsfonts} % математический пакет, математические шрифты
\usepackage{textcomp}
\usepackage[noend]{algorithmic}
\usepackage[ruled]{algorithm}
\usepackage{lipsum}
\usepackage{indentfirst}
\usepackage{babel}
\usepackage{pgfplots}
\usepackage{setspace}
\usepackage{xcolor}
\usepackage{hyperref}
\usepackage{subfigure}

\setcounter{secnumdepth}{5}
\setcounter{tocdepth}{5}
\newcommand\simpleparagraph[1]{%
  \stepcounter{paragraph}\paragraph*{\theparagraph\quad{}#1}}
\usepackage{listings}
% \usepackage{xcolor}
%\usepackage{minted}

\lstset { %
     language=C++,
     backgroundcolor=\color{black!5}, % set backgroundcolor
     basicstyle=\footnotesize,% basic font setting
}


\linespread{1.0} 
\setlength{\parindent}{2.4em}
\setlength{\parskip}{0.1em}

\pgfplotsset{compat=1.9}
\pgfplotsset{model/.style = {blue, samples = 100}} 
\pgfplotsset{experiment/.style = {red}}

\theoremstyle{plain}
\binoppenalty=10000

\newtheorem{theorem}{Theorem}[section] % theorem

\theoremstyle{definition}
% \newtheorem{definition}{Определение}[subsection]
\newtheorem{definition}{Definition}[subsection]

\theoremstyle{remark}
% \newtheorem{remark}{Замечание}[section]

% \newtheorem{corollary}{Следствие}

% \newtheorem{proposition}{Proposition}

% \newtheorem{example}{Пример}

% \newtheorem{lemma}{Лемма}[section]

\renewcommand*{\proofname}{Proof}

\graphicspath{ {./images/} }


% \usepackage{amsmath,amsfonts,amssymb, setspace}  % Разнообразные математические команды и значки
% \usepackage{indentfirst}     % Отступ в первом абзаце

% \pagestyle{empty}
\usepackage[left=2.5cm, right=1.5cm, top=2.5cm, bottom=2.5cm]{geometry}
\usepackage[medium]{titlesec}
\usepackage{graphicx}
% \graphicspath{ {./images/} }

\begin{document}

	\begin{titlepage} 
		\begin{center}
		\textbf{}\\[2.0cm]
		\LARGE FEDERAL STATE AUTONOMOUS EDUCATIONAL INSTITUTION OF HIGHER EDUCATION \\[0.5cm]
		\Large ITMO UNIVERSITY \\[3cm]
		\LARGE Report\\
		\Large MPI. Assignments $6-8$ \\
		\Large Parallel algorithms for the analysis and synthesis of data \\[4cm]


		\begin{flushright}
		Performed by\\
		Aleksandr Shirokov\\
		J4133c\\
		Accepted by\\
		Petr Andriushchenko

		Deadline: 20.12.21
		\end{flushright}

		\vfill 

		{\Large {St. Petersburg}} \par
		{\Large {2021}}
		\end{center} 
	\end{titlepage}

\tableofcontents
\newpage


\section{Assignments}

\subsection{Assignment 6.}

\subsubsection{Formulation of the problem}

\begin{enumerate}
	\item Compile the example \textsc{Assignment6.c} in detail, run it and explain it.
	\item Transform the program using the \textsc{MPI\_TAG} field of the status structure in the 
condition.
\end{enumerate}

\subsubsection{Example of launch parameters and output. Detailed description of solution}

Code for \textbf{assignment 6} is \href{https:\//github.com/aptmess/parallel_algorithms/blob/master/HT/hw_mpi/Assignment6.c}{here}.

Compilation example: \textsc{mpic++ -o ./cpf/6.o Assignment6.c}

Launch example: \textsc{mpirun --oversubscribe -np 4 ./cpf/6.o}

\begin{center}
\includegraphics[scale=0.7]{6.1.png}

There could be only two results of program output
\end{center}

Let's move to the the code and explain how it works.

\begin{center}
\includegraphics[scale=0.95]{6.code.png}

Assignment6 code
\end{center}

Firstly there is an initialization of parallel part using \textsc{MPI\_Init}, after if rank of process is $1$ then the int $1$ will be send as a message and if rank of process is $2$, then the float value $2.0$ will be send as message. After we are going to main process $0$ logic: 

\begin{itemize}
	\item \textsc{MPI\_Probe} this function is waiting for message from any process with $\operatorname{msgtag}=5$ and wouldn't go next if the message doesn't come to process $0$. Let's make it clear - function only understand that message come to process, but doesn't get it.
	\item After that if \textsc{status.MPI\_SOURCE == 1} so if first was message from process $1$ then there is a print message that $1$st process's message was quicklier, else - that the second was quicklier and the value from second process will be displayed first.
\end{itemize}

After I have transformed the problem using \textsc{MPI\_TAG} field. Here are results:

\begin{center}
\includegraphics[scale=0.65]{6.2.png}

Results are the same. Take a look at code
\end{center}

Code for \textbf{assignment 6.1} is \href{https:\//github.com/aptmess/parallel_algorithms/blob/master/HT/hw_mpi/Assignment6.1.c}{here}.

Compilation example: \textsc{mpic++ -o ./cpf/6.1.o Assignment6.1.c}

Launch example: \textsc{mpirun --oversubscribe -np 4 ./cpf/6.1.o}

\begin{center}
\includegraphics[scale=0.7]{6.2.code.png}

Assignment6 part II code
\end{center}

Everything is more or less the same, but now we are expecting any tag in \textsc{MPI\_Probe} function and processes $1$ and $2$ has different tags ($5$ and $4$) and condition is also have changed (\textsc{status.MPI\_TAG}). Program works correctly.



\newpage
\subsection{Assignment 7.}

\subsubsection{Formulation of the problem}

\subsubsection{Example of launch parameters and output. Detailed description of solution}

Code for \textbf{assignment 7} is \href{https:\//github.com/aptmess/parallel_algorithms/blob/master/HT/hw_mpi/Assignment7.c}{here}.

Compilation example: \textsc{mpic++ -o ./cpf/7.o Assignment7.c}

Launch example: \textsc{mpirun --oversubscribe -np 4 ./cpf/7.o}

% \begin{center}
% 		\includegraphics[scale=0.5]{3.1.png}
% \end{center}

Let's move to the the code and explain how it works.

% \begin{center}
% \includegraphics[scale=0.75]{3.code.png}

% Assignment6 code
% \end{center}

Explain. 

\newpage
\subsection{Assignment 8.}

\subsubsection{Formulation of the problem}



\subsubsection{Example of launch parameters and output. Detailed description of solution}

Code for \textbf{assignment 8} is \href{https:\//github.com/aptmess/parallel_algorithms/blob/master/HT/hw_mpi/Assignment8.c}{here}.

Compilation example: \textsc{mpic++ -o ./cpf/6.o Assignment6.c}

Launch example: \textsc{mpirun --oversubscribe -np 4 ./cpf/6.o}

% \begin{center}
% 		\includegraphics[scale=0.5]{3.1.png}
% \end{center}

Let's move to the the code and explain how it works.

% \begin{center}
% \includegraphics[scale=0.75]{3.code.png}

% Assignment6 code
% \end{center}

Explain. 



\subsection{Appendix}

The link to the sourse code which is placed on my \href{https://github.com/aptmess/parallel_algorithms}{github}.


\end{document}